
% Commands for running this example:
% 	 xelatex main
% 	 bibtex8 -W -c cp1256fa main
%      xindy -L persian -C utf8 -M texindy main
% 	 xelatex main
% 	 xelatex main
% End of Commands

%        نمونه پایان‌نامه آماده شده با استفاده از کلاس IUST-Thesis، نگارش 0.6
% 		محمود امین‌طوسی، دانشگاه تربیت معلم سبزوار، http://profsite.sttu.ac.ir/mamintoosi/
% 		گروه پارسی‌لاتک  http://www.parsilatex.com
%        این نسخه، بر اساس نسخه‌ 0.4 از کلاس Tabriz_Thesis آقای وحید دامن‌افشان آماده شده است. http://damanafshan.tk
%        
%        تغییرات:
%        نسخه 0.6:
%        اصلاح مشکل بسته subfig 
%----------------------------------------------------------------------------------------------
%        اگر قصد نوشتن پروژه کارشناسی را دارید، در خط زیر به جای msc، کلمه bsc و اگر قصد نوشتن پروژه دکترا
%        را دارید، کلمه phd را قرار دهید. کلیه تنظیمات لازم، به طور خودکار، اعمال می‌شود.

%        اگر مایلید پایان‌نامه شما دورو باشد به جای oneside  در خط زیر از twoside استفاده کنید
\documentclass[oneside,openany,msc]{IUST-Thesis}

% مشخصات پایان‌نامه را در فایلهای faTitle و enTitle وارد نمایید.

%       فایل commands.tex را مطالعه کنید؛ چون دستورات مربوط به فراخوانی بسته زی‌پرشین 
%       و دیگر بسته‌ها و ... در این فایل قرار دارد و بهتر است که با نحوه استفاده از آنها آشنا شوید.
\input{commands}

\begin{document}


\pagenumbering{harfi}
\input{Sections/faTitle}
\tableofcontents

\newpage
\listoffigures \newpage
\listoftables  \newpage
\addcontentsline{toc}{chapter}{\listalgorithmname}
\listofalgorithms \newpage
\input{Sections/acronyms}

\pagestyle{fancy}
% اگر شما فصل اول  خود را در فایلی به جز chapter1 همراه با این کلاس نوشته‌اید باید چندخط اول chapter1 را در فایل خود کپی کنید.
\include{Sections/intro}			% فصل اول: مقدمه
\include{Sections/latexIntro}		% فصل دوم: آشنایی مقدماتی با لاتک

% مراجع
\pagestyle{empty}
{
\onehalfspacing
\bibliographystyle{acm-fa}%{chicago-fa}%{plainnat-fa}%
\bibliography{MyReferences}
}

\pagestyle{fancy}

\appendix                           %فصلهای پس از این قسمت به عنوان ضمیمه خواهند آمد.
% اگر شما پیوست اول  خود را در فایلی به جز appendix1 همراه با این کلاس نوشته‌اید باید چندخط اول appendix1 را در فایل خود کپی کنید.
\include{Sections/appendix1}		% پیوست اول: مدیریت مراجع در لاتک
\include{Sections/appendix2}

%\baselineskip=.75cm
\onehalfspacing
\include{Sections/dicfa2en}
\include{Sections/dicen2fa}
\printindex
% !TeX root=main.tex
% در این فایل، عنوان پایان‌نامه، مشخصات خود و چکیده پایان‌نامه را به انگلیسی، وارد کنید.

%%%%%%%%%%%%%%%%%%%%%%%%%%%%%%%%%%%%
\baselineskip=.6cm
\begin{latin}
\latinuniversity{Tarbiat Modares University}
\latinfaculty{Electrical and Computer Engineering Department}
\latinsubject{Computer Engineering }
\latinfield{Artificial Intelligence and robotics}
\latintitle{Writing projects, theses and dissertations using LaTex for TMU}
\firstlatinsupervisor{Dr. Mohammad saniee abadeh}
%\secondlatinsupervisor{Second Supervisor}
\firstlatinadvisor{Dr. Mehdi ro'ayaei ardakani}
%\secondlatinadvisor{Second Advisor}
\latinname{Arash}
\latinsurname{Jahdkaran}
\latinthesisdate{June 2022}
\latinkeywords{Writing Thesis, Template, \LaTeX, \XePersian}
\en-abstract{
This thesis studies on writing projects, theses and dissertations using IUST-Thesis Class. It ...
}
\latinfirstPage
\end{latin}

\label{LastPage}

\end{document}